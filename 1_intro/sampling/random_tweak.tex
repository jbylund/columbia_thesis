Random tweak, like CCD, is a method of producing and sampling closed loop conformations.
It begins in much the same way as CCD, by randomizing $\phi$ and $\psi$ dihedral angles of the loop backbone.
Random tweak seeks to close the loop while retaining dihedral angles as close to the randomized starting structure as possible.
By adjusting each dihedral only a small amount at a time and staying in the region where $sin(\theta) \approx \theta$, it is possible to formulate a set of linear equations to solve for a set of $\Delta\theta_{i}$, which minimizes the distance between the crystal position of the atom to be closed and the random position.
Because the assumption $sin(\theta) \approx \theta$ only holds for small $\theta$, the maximum change in angle is limited to 10 degrees in the original implementation of the random tweak algorithm.
Because almost all structures predicted using the random tweak or cyclic coordinate descent produce closed loops, a much smaller fraction of time is spent sampling loops that do not satisfy the closure criteria, making these algorithms very efficient \cite{fine1986predicting,shenkin1987predicting}.
