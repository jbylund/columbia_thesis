\subsection*{Qualititave Measures of Prediction Quality}
\label{subsec:results_quality}

For side chain prediction experiments, 85.2\% of side chain prediction conformations (9406 of 11030 total) predicted with the new cell based solvation model are within 0.2 angstrom heavy atom RMSD of the prediction using the naive implementation.
In other metrics, the quality of prediction is comparable between the two solvent models. 
Median side chain heavy atom RMSD is 0.567 and 0.558 angstroms for the cell based method and the non-cell based method, respectively.
Average RMSD to the crystal structure is similarly close, 1.11 angstroms for both methods, with 79.9\% of side chain predictions within 2 angstroms RMSD of the native using the cell based model and 79.4\% within two angstroms using the naive approach.
Of side chains which are predicted differently by the two implementations there is no correlation between solvation model and prediction quality.
The distribution of side chain predictions with respect to RMSD to native is also indistinguishable between the two methods of computing the solvation term.

Data for energy calculations is not presented here because it is identical in every case.
This is expected, given that the two models represent two methods of computing the same quantity.
Thus, on the whole, prediction accuracy of the hash based model is comparable with the old implementation.

\subsection*{Performance Improvement}
\label{subsec:performance_improvement}
The principal goal of the hash based approach is to improve the performance of the implicit solvent models. 
Thus, the key metric of performance improvement is the speedup over the previous implementation.
Energy computations were found to be from 1.6 to 2.5 times as fast, and the trend indicates that even larger improvements would be obtained in larger system.
This sort of experiment represents a ``best case'' for the expected performance increase of a hash based solvent, as they represent a minimum amount of time spent on other parts of the experiment.
Implicit solvent calculations, and energy calculations in general, compose a smaller fraction of time in simultaneous side chain prediction therefore the observed performance improvement is less than that of energy calculations.
The observed performance increase in this sort of experiment is still on approximately 20\%.

\begin{figure}[H]
\begin{center}
\includegraphics[width=0.8\textwidth]{figures/energy_calculation_timings.png}
\caption{Energy computations using a grid based method yields approximately a three times performance improvement, though in the case of some very small structures it is possible that the overhead introduced by maintaining the grid structure outweighs the improvement.}
\label{fig:ddr}
\end{center}
\end{figure}

\begin{table}[H]
\centering
\label{table:energy_timings}
\begin{tabular}{|c|c|c|}
\hline
PDB id	& Naive Method	& Cell Based Method	\\
\hline
1F5Z	&   201.75	&   96.37	\\
1H2V	&   98.31	&   55.32	\\
1HRD	&   228.38	&   110.12	\\
1M1Z	&   147.83	&   71.28	\\
1M9X	&   282.35	&   113.04	\\
1O60	&   172.82	&   82.34	\\
1R0V	&   210.35	&   90.0    \\
1XMP	&   213.25	&   111.25	\\
2E3Z	&   141.11	&   78.79	\\
2H6U	&   138.25	&   65.07	\\
2OU1	&   104.93	&   57.24	\\
2XI9	&   71.68	&   45.73	\\
3AMD	&   288.58	&   116.52	\\
3DEL	&   133.62	&   76.48	\\
3E1E	&   183.8	&   83.8	\\
3FGN	&   120.38	&   57.75	\\
3HHP	&   195.25	&   97.58	\\
4GVR	&   136.35	&   84.78	\\
\hline
\end{tabular}
\caption{The specific timings for a series of energy computations presented in Figure \ref{fig:ddr}.
These represent a ``best case'' scenario, as the majority of time in these experiments is spent computing the solvent contribution.}
\end{table}
% Would it be useful in this table to add a column that shows %improvement?



