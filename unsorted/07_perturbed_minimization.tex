\section{Perturbed Minimization}
\label{section:unsorted/perturbed_minimization}

The specific minimization method used in PLOP is a modified version of the TNPACK, truncated Newton minimization package \cite{schlick1992tnpack,schlick1992tnpack2}.
This has been shown to have very good performance on high dimensional minimization problems.
However, as the energy surface of large structures is very rough, having a large number of local minima, simple minimization methods tend to be sensitive to initial conformations, and occasionally small changes in starting conformations, or internal parameters can have a large effect on the final minimized energy, \ref{figure:sensitive_initial_conditions}.
One such approach to many minima problems has been simulated annealing \cite{kirkpatrick1983optimization,vcerny1985thermodynamical}.
However, this performs significantly more sampling, and this requires appreciably more time than a simple minimization method.
Therefore it is desirable to find a minimization method, which is both somewhat tolerant to local minima and insensitive to initial conformations and quickly arrives upon a solution.
We have performed some inital experiments with such a method.
\begin{figure}
\centering
\includegraphics[width=0.4\textwidth,height=0.4\textheight,keepaspectratio]{figures/sensitive_initial_conditions.png}
\caption{}
\label{figure:sensitive_initial_conditions}
\end{figure}

Our method, perturbed minimization, combines iterative minimization from slightly varied starting conformations with an adaptive stopping rule in order to minimize time spent sampling after finding a reasonably minima.
The stopping rule, inspired by solutions to the classic game theory secretary problem balances continuing to sample while new lower energy conformations are likely to be found, with efficient stopping after sampling ceases to be productive \cite{freeman1983secretary,chow1964optimal}.

\begin{figure}
\centering
\includegraphics[width=0.5\textwidth,height=0.5\textheight,keepaspectratio]{figures/perturbed_minimization_flowchart.png}
\caption{}
\label{figure:perturbed_minimization_flowchart}
\end{figure}

Specifically, the structure is perturbed at the before applying any minimization.
The new perturbed structure is minimized, and if the energy is lower than the current best energy the new low energy conformation along with the energy are saved.
The structure is then reverted to its initial conformation and the procedure is repeated until more than a specified number of steps are performed without finding a new lowest energy structure.
This is effectively the successive non-candidate heuristic solution to the secretary problem applied in reverse (as in this case it is possible to ``call back'' a candidate).
This procedure is described graphically in \ref{figure:perturbed_minimization_flowchart}
This method has been applied to a number of test sets with promising initial results, though it is not clear at the moment in which circumstances this sort of sampling may be necessary.
