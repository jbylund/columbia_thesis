%1BRSa
\begin{table}[H]
\centering
\label{table:1brs_a_results}
\begin{tabular}{|c|c|c|}
\hline
Residue & \ddg\ calculated & \ddg\ experimental \\
\hline
native & 0 & 0 \\
27 & 23.82 & 5.4 \\
54 & -1.37 & -0.8 \\
58 & 9.09 & 3.1 \\
59 & 11.58 & 5.2 \\
60 & 11.15 & -0.2 \\
73 & -7.28 & 2.8 \\
87 & 40.32 & 5.5 \\
102 & -16.83 & 6 \\
\hline
\end{tabular}
\caption{}
\end{table}

\begin{table}[h]
\centering
\label{table:1fcc_side_pred_rmsd}
\begin{tabular}{|c|c|c|}
\hline
Residue & Amino Acid & RMSD \\
\hline
A:27 & LYS & 1.213 \\
A:54 & ASP & 0.920 \\
A:58 & ASN & 0.121 \\
A:59 & ARG & 0.421 \\
A:60 & GLU & 0.138 \\
A:73 & GLU & 0.994 \\
A:87 & ARG & 0.297 \\
A:102 & HIS & 0.276 \\
\hline
\end{tabular}
\caption{RMSD of mutated side chains in barnase, in a barnase-barstar complex (chain A of PDBid 1BRS), during the mutation scanning experiments.}
\end{table}


\begin{figure}[h]
  \centering
  \includegraphics[width=0.65\textwidth]{figures/1brs_barnase_barstar.png}
  \caption{
Computed versus experimental \ddg\ binding for 8 alanine mutations in the Barstar-Barnase binding pair.
Crystal structure used for computations was 1BRS.
Specific amino acids mutated were residues 27, 54, 58, 59, 60, 73, 87, and 102, all of chain A.
Experimental binding affinity taken from \protect\cite{thorn2001asedb}.
            }
\end{figure}

\begin{figure}[h]
  \centering
  \includegraphics[width=0.65\textwidth,height=0.3\textheight,keepaspectratio]{figures/mutation_side_chain_images/1brs_chain_a_resid_58.png}
  \caption{Crystal, colored by atom, and predicted, magenta, side chain conformations for barnase, asparagine 58 of 1BRS.
The predicted and crystal conformations are almost identical, differing by only 0.121 angstroms, or less than the resolution of the crystal structure.}
  \label{figure:computational_mutation_scanning/1brs_a_58}
\end{figure}

\begin{figure}[h]
  \centering
  \includegraphics[width=0.65\textwidth,height=0.3\textheight,keepaspectratio]{figures/mutation_side_chain_images/1brs_chain_a_73.png}
  \caption{Crystal, colored by element, and predicted, gray, side chain conformations of glutamic acid 73 of barnase, chain A of PDBid 1BRS.
The two conformations differ by 0.993 angstrom RMSD, which is generally considered a successful sidechain prediction, though towards the upper range of a successful prediction.}
  \label{figure:computational_mutation_scanning/figname}
\end{figure}
\clearpage

% 1BRSd
\begin{table}[h]
\centering
\begin{tabular}{|c|c|c|}
\hline
Residue & \ddg\ calculated & \ddg\ experimental \\
\hline
native & 0 & 0 \\
29 & 121.07 & 3.1 \\
35 & 118.37 & 5.2 \\
39 & 118.09 & -0.2 \\
42 & 141.09 & 2.8 \\
74 & 107.92 & 5.5 \\
78 & 112.14 & 6 \\
\hline
\end{tabular}
\caption{Calculated and experimental \ddg\ for mutating given residues of barstar (chain D of structure 1BRS) to alanine.
Experimental values taken from \protect\cite{thorn2001asedb}.}
\label{table:1BRSd_results}
\end{table}

\begin{table}[h]
\centering
\begin{tabular}{|c|c|c|}
\hline
Residue & Amino Acid & RMSD \\
\hline
D:29 & TYR & 0.121 \\
D:35 & ASP & 0.098 \\
D:39 & ASP & 0.335 \\
D:42 & THR & 0.114 \\
D:76 & GLU & 0.397 \\
D:80 & GLU & 1.804 \\
\hline
\end{tabular}
\caption{RMSD of mutated side chains in barstar, in a barnase-barstar complex (chain D of PDBid 1BRS), during the mutation scanning experiments.}
\label{table:1BRSd_rmsd}
\end{table}


\begin{figure}[h]
  \centering
  \includegraphics[width=0.65\textwidth]{figures/1brs_barstar_barnase.png}
  \caption{
Computed versus experimental \ddg\ binding for 6 alanine mutations in the Barstar-Barnase binding pair.
Crystal structure used for computations was 1BRS \protect\cite{buckle1994protein}.
Specific amino acids mutated were residues 29, 35, 39, 42, 74, and 78, all of chain D.
Experimental binding affinity taken from \protect\cite{thorn2001asedb}.
            }
\end{figure}

\begin{figure}[h]
  \centering
  \includegraphics[width=0.65\textwidth,height=0.3\textheight,keepaspectratio]{figures/mutation_side_chain_images/1brs_all.png}
  \caption{Distribution of 6 mutated residues, shown in magenta, on the interface surface of barstar, 1BRS chain D.
Five of the six residues are less than 0.4 angstroms RMSD to the crystal structure.
The only exception is, glutamic acid 80, shown in the upper left of this figure, and also \protect\cite{figure:computational_mutation_scanning/1brs_d_80}.}
  \label{figure:computational_mutation_scanning/figname}
\end{figure}

\begin{figure}[h]
  \centering
  \includegraphics[width=0.65\textwidth,height=0.3\textheight,keepaspectratio]{figures/mutation_side_chain_images/1brs_chain_d_35.png}
  \caption{Crystal, colored by atom, and predicted, magenta, side chain conformations for barstar, chain D of PDBid 1BRS.
The distance to the crystal structure is only 0.098 angstroms, or nearly identical.}
  \label{figure:computational_mutation_scanning/figname}
\end{figure}

\begin{figure}[h]
  \centering
  \includegraphics[width=0.65\textwidth,height=0.3\textheight,keepaspectratio]{figures/mutation_side_chain_images/1brs_chain_d_80.png}
  \caption{Glutamic acid 80 is the only residue on chain D, barstar, of the barnase-barstar complex which was not predicted within 0.4 angstroms of the crystal coordinates during the mutation scanning experiments.
The difference between these two conformations is 1.804 angstroms, which while sometimes considered a ``successful'' prediction, is not sufficiently close to generate the same interactions, making it difficult to accurately predict binding affinities.}
  \label{figure:computational_mutation_scanning/1brs_d_80}
\end{figure}
\clearpage

% 1DVF
\begin{table}[h]
\centering
\label{table:1dvf_results}
\begin{tabular}{|c|c|c|}
\hline
Residue & \ddg\ calculated & \ddg\ experimental \\
\hline
native & 0 & 0 \\
30 & -42.93 & 0.9 \\
32 & -44.68 & 1.8 \\
52 & -42.6 & 4.2 \\
54 & -28.29 & 4.3 \\
56 & -37.5 & 1.2 \\
58 & -41.91 & 1.6 \\
98 & -34.38 & 4.2 \\
99 & -30.51 & 1.9 \\
100 & -60.9 & 2.8 \\
101 & -37.84 & 4 \\
\hline
\end{tabular}
\caption{Calculated and experimental \ddg\ for mutating given residues of anti-idiotopic antibody (chain A of structure 1BRS) to alanine.
Experimental values taken from \protect\cite{thorn2001asedb}.}
\end{table}

\begin{table}[h]
\centering
\begin{tabular}{|c|c|c|}
\hline
Residue & Amino Acid & RMSD \\
\hline
B:30 & THR & 0.056 \\
B:32 & TYR & 0.412 \\
B:52 & TRP & 0.391 \\
B:54 & ASP & 0.442 \\
B:56 & ASN & 0.238 \\
B:58 & ASP & 0.159 \\
B:98 & GLU & 0.182 \\
B:99 & ARG & 1.137 \\
B:100 & ASP & 2.577 \\
B:101 & TYR & 0.340 \\
\hline
\end{tabular}
\caption{RMSD of mutated side chains in 1DVF, anti-hen-egg-white lysozyme antibody (D1.3) complexed with an anti-idiotopic antibody (E5.2), during the mutation scanning experiments.}
\label{table:1DVF_rmsd}
\end{table}


\begin{figure}[h]
    \centering
  \includegraphics[width=0.65\textwidth]{figures/1dvf.png}
  \caption{
Computed versus experimental \ddg\ binding for 10 alanine mutations in the anti-hen-egg-white lysozyme antibody (D1.3) anti-idiotopic antibody (E5.2) complex.
Crystal structure used for computations was 1DVF \protect\cite{braden1996crystal}.
Specific amino acids mutated were residues 30, 32, 52, 54, 56, 58, 98, 99, 100, and 101, all of chain A.
Experimental binding affinity taken from \protect\cite{thorn2001asedb}.
            }
\end{figure}

\begin{figure}[h]
  \centering
  \includegraphics[width=0.65\textwidth,height=0.3\textheight,keepaspectratio]{figures/mutation_side_chain_images/1dvf_chain_b_100.png}
  \caption{The caption.}
  \label{figure:computational_mutation_scanning/figname}
\end{figure}

\begin{figure}[h]
  \centering
  \includegraphics[width=0.65\textwidth,height=0.3\textheight,keepaspectratio]{figures/mutation_side_chain_images/1dvf_chain_b_30_and_32.png}
  \caption{The caption.}
  \label{figure:computational_mutation_scanning/figname}
\end{figure}
\clearpage

% 1FCC
\begin{table}[!h]
\centering
\begin{tabular}{|c|c|c|}
\hline
Residue & \ddg\ calculated & \ddg\ experimental \\
\hline
native & -0.18 & 0 \\
25 & -4.55 & 0.24 \\
27 & 19.8 & 4.9 \\
28 & 26.37 & 1.3 \\
31 & 18.82 & 3.5 \\
35 & -6.31 & 2.4 \\
40 & -8.51 & 0.3 \\
42 & -5.56 & 0.4 \\
43 & 12.0 & 3.8 \\
\hline
\end{tabular}
\caption{Calculated and experimental \ddg\ for mutating given residues of Fc domain of human IgG (chain A of structure 1FCC) to alanine.
Experimental values taken from \protect\cite{thorn2001asedb}.}
\label{table:1FCC_results}
\end{table}

\begin{table}[!h]
\centering
\begin{tabular}{|c|c|c|}
\hline
Residue & Amino Acid & RMSD \\
\hline
C:25 & THR & 0.170 \\
C:27 & GLU & 0.403 \\
C:28 & LYS & 0.543 \\
C:31 & LYS & 0.594 \\
C:35 & ASN & 0.527 \\
C:40 & ASP & 1.292 \\
C:42 & GLU & 2.994 \\
C:43 & TRP & 0.371 \\
\hline
\end{tabular}
\caption{RMSD of mutated side chains in 1FCC, C2 fragment of streptococcal protein G in complex with the Fc domain of human IgG, during the mutation scanning experiments.}
\label{table:1FCC_rmsd}
\end{table}


\begin{figure}[h]
    \centering
  \includegraphics[width=0.65\textwidth]{figures/1fcc.png}
  \caption{
Computed versus experimental \ddg\ binding for 8 alanine mutations in  binding pair.
Crystal structure used for computations was 1FCC.
Specific amino acids mutated were residues 25, 27, 28, 31, 35, 40, 42, and 43, all of chain A.
Experimental binding affinity taken from \protect\cite{thorn2001asedb}.
          }
\end{figure}

\begin{figure}[h]
  \centering
  \includegraphics[width=0.65\textwidth,height=0.3\textheight,keepaspectratio]{figures/mutation_side_chain_images/1fcc_27_31_43.png}
  \caption{The caption.}
  \label{figure:computational_mutation_scanning/figname}
\end{figure}

\begin{figure}[h]
  \centering
  \includegraphics[width=0.65\textwidth,height=0.3\textheight,keepaspectratio]{figures/mutation_side_chain_images/1fcc_27.png}
  \caption{The caption.}
  \label{figure:computational_mutation_scanning/figname}
\end{figure}

\begin{figure}[h]
  \centering
  \includegraphics[width=0.65\textwidth,height=0.3\textheight,keepaspectratio]{figures/mutation_side_chain_images/1fcc_31.png}
  \caption{The caption.}
  \label{figure:computational_mutation_scanning/figname}
\end{figure}

\begin{figure}[h]
  \centering
  \includegraphics[width=0.65\textwidth,height=0.3\textheight,keepaspectratio]{figures/mutation_side_chain_images/1fcc_43.png}
  \caption{Native, colored by element, and predicted, magenta side chain conformation for tryptophan 43 of 1FCC.
The root mean square distance between the two side chains is 0.371 angstroms.}
  \label{figure:computational_mutation_scanning/1fcc_43}
\end{figure}

\begin{figure}
    \centering
    \begin{subfigure}[b]{0.3\textwidth}
        \centering
        \includegraphics[width=\textwidth,height=\textheight,keepaspectratio]{figures/mutation_side_chain_images/in_pocket_out_of_plane.png}
        \caption{}
        \label{figure:mutation_side_chain_images/in_pocket_out_of_plane}
    \end{subfigure}
    \hspace{0.1\textwidth}
    \begin{subfigure}[b]{0.3\textwidth}
        \centering
        \includegraphics[width=\textwidth,height=\textheight,keepaspectratio]{figures/mutation_side_chain_images/in_pocket_in_plane.png}
        \caption{}
        \label{figure:mutation_side_chain_images/in_pocket_in_plane.png}
    \end{subfigure}
    \caption{The pocket of tryptophan 43 of 1FCC.  
Because of conformation of the neighboring protein structure this residue has very little conformational freedom, and any prediction which successfully locates the sidechain in the pocket will be reasonably close to the native state.
The conformation predicted in these experiments was very similar, 0.371 angstroms, and is depicted superimposed with the native in figure \protect\ref{1fcc_43}.}
    \label{figure:1fcc_43_pocket}
\end{figure}


