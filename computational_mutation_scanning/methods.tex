\subsection{Entropy-Enthalpy Compensation}
Some computational alanine scanning experiments explicitly compute or approximate the entropic contribution to the change in the free energy of binding \cite{hao2010computational,guerois2002predicting}. 
However, other models have achieved good agreement with experimental data while assuming these effects are either accounted for by the correlation between entropy and enthalpy for small changes in protein structure \cite{sharp2001entropy} or to be small relative the entropic changes \cite{kortemme2004computational}.
PLOP has not made use of entropic contributions to free energy and has in many cases achieved good agreement with experimental data, so in these experiments it is assumed that contributions due to entropy are small relative entropic contributions.

\subsection{General Mutation Screening}
The generalized mutation screening method implemented in PLOP allows efficient evaluation of a large number of possible mutations to proteins.
It accepts as input a set of possible mutations for each residue, or a set of possible mutations for a set of residues.
For instance, tryptophan, tyrosine, and arginine are overrepresented in hot spot residues \cite{hu2000conservation}, so it may be desirable to consider all mutations in which a set of residues are either left at their native identity or replaced with one of these residues.
If desired, the user can also set bounds for the minimum and maximum number of simultaneous mutations allowed.
The residues that will be mutated are referred to as {\it free} residues, as the conformations of the other residues are held fixed throughout the entire process.
While the residues are still in their native states, the structure is subjected to some sort of sampling.
This is done in order to prevent bias towards predictions that will later be sampled in the same fashion.
In the present implementation this consists of predicting the conformation of each free residue and minimizing those residues.

In this side chain sampling, for each free residue, the side chain is initially replaced with a random conformation from a high resolution rotamer library screened for steric clashes with the static part of the protein.
The free residues are then examined sequentially, replacing each with the lowest energy conformation present in the rotamer library.
This replacement process is continued until the termination condition is met, which is that two or fewer residues are replaced by lower energy conformations during the replacement stage.
Five iterations of this procedure, from randomization to a static conformation, are performed and the most frequently selected conformation is chosen for each amino acid \cite{jacobson2002force,jacobson2002role}.

In the mutation stage, each free residue is first updated to its new chemical identity, possibly remaining in the native state, side chains are replaced with the side chain of the desired amino acid, with the corresponding updates to the bond, angle, torsion, and 1-4 interactions.
The conformations of free residues are then re-predicted using the same side chain prediction algorithm described for the native conformation.

\subsection{Alanine Scanning Experiments}
Three protein complexes (PDBids: 1FCC, 1BRS, and 1DVF) with both experimental data for binding affinity and crystal structures, were identified using the ASEdb \cite{thorn2001asedb,sauer1995crystal,buckle1994protein,braden1996crystal}.
Protonation states and locations of polar hydrogens were assigned for all residues as in \cite{li2007assignment}.
A crystal context was built for each structure using symmetry data determined by experiment.
For each mutation represented in the alanine scan database, a single residue was mutated to alanine and this side chain prediction was repeated.
The resulting structures were examined for side chain conformation agreement with crystal structures and the change in binding free energy to native was recorded. 
