\subsection{General Mutation Screening}
The generalized mutation screening method implemented in PLOP allows efficient evaluation of a large number of possible mutations.
It accepts as input a set of possible mutations for each residue, or a set of possible mutations for a set of residues.
For instance tryptophan, tyrosine and arginine are overrepresented in hot spot resiudes \cite{hu2000conservation}, so it may be desirable to consider all mutations in which a set of residues are either left at their native identity or replaced with one of these residues.
If desired the user can also set bounds for the minimum and maximum number of simultaneous mutations allowed.
While the residues are still in their native states the conformation of each residue which will be mutated is re-predicted and minimized.
This is done in order to prevent bias towards predicted states which will later be predicted in the same fashion.
For each residue for which mutations are allowed the sidechain 
These side chain predictions were performed by choosing at random from a high resolution rotamer library an initial conformation.
Each residue was then replaced sequentially by the lowest energy conformation present in the rotamer library.
This procedure was continued until all residues were static.
Five iterations of this procedure, from randomization to a static conformation, were performed for each structure \cite{jacobson2002force,jacobson2002role}.




\subsection{Alanine Scanning Experiments}
Three protein complexes, 1FCC, 1BRS, and 1DVF, with both experimental data for bindinging affinity and crystal structures were identified using the ASEdb \cite{thorn2001asedb}.
Protonation states and locations of polar hydrogens were assigned for all residues as in \cite{li2007assignment}.
A crystal context was built for each structure using symmetry data determined by experiment.

For each mutation represented in the alanine scan database single residue was mutated to alanine and this side chain prediction was repeated.
The resulting structures were examined side chain conformation agreement with crystal structures and the change in binding free energy to native was recorded. 
