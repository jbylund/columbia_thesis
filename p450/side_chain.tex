Side chain moves.
Several types of side chain motions were implemented in PLOP.
In all cases, they are defined in a such a way that they can be applied to both ligands and proteins.
The same atomic overlap screening function implemented with the rigid body Monte Carlo was implemented with the side chain torsional moves.
a. Random torsion angle moves: The first type of move that was implemented is random movement of torsional chi angles.
For small torsion moves, a random perturbation of the angle of +/- X is made, where X is a random number with user defined magnitude.
For large torsion moves, for each torsion angle that is changed, a random angle is selected in the form 60*Y +/- X, where Y = 1 through 5, and X is the same random number for the small torsion moves.
The large move was introduced since positions at the top of rotamer barriers are relatively unlikely to be selected, and efficiency thus can be improved by focusing on the more probable moves.
The ratio of small to large torsion moves can be used-adjusted, as can the ratio of probabilities of changing all the torsions in a randomly selected side chain versus changing only one single (randomly selected) torsion among all the free torsions in the simulation can be set as a user-defined parameter.
Rotamer side chain moves: A second type of torsional samples implemented is random selection of a new rotamer state for the entire side chain, plus an optional user defined small noise term for each torsion in the rotamer state.
A database of protein rotamer states obtained from crystallographic data are already a part of PLOP \cite{xiang2001extending} Rotamer libraries for ligands are generated by examining all possible side chain conformations at 10 degree resolution and screening this set for steric clashes.
A Monte Carlo move in this case represents a choice of a new torsional rotamer state for the entire side chain.
Monte Carlo moves based on torsional states cannot lead to correct equilibrium distributions, as transitions from non-rotamer states to rotamer states are defined, but not reverse transitions, upsetting detailed balance.
However, a pretabulated rotamer state is more likely to be low energy than a randomly generated torsional state, and thus allows for more diverse conformational searching.
Correlated torsional moves: Most torsional rearrangements of the side chains in the core of proteins are highly correlated because of the density.
In order to attempt to include correlated torsional motion, at each step we examine the distance between all pairs of beta carbons in the ligands that are free to move.
At each step, for the set of side chains that are free to move, clusters where beta carbons are all mutually within a user-specified distance are identified.
This process takes a trivial amount of time compared to an energy evaluation, so does not slow the simulation at all.
Then, with user specified probabilities, clusters of different sizes are selected for the torsional moves, either with random side chain moves, or rotamer selection moves.
By selecting only clusters where all residues are mutual neighbors, detailed balanced is observed for simulations where accurate equilibrium sampling is desired.
By varying the dihedral angles of the rotatable bonds, IDSite uses side chain MC moves in PLOP to sample the selected side-chain conformations of the protein and of the ligand.
Up to three close residues (C beta distance within 6 angstroms) are allowed to rotate collectively, but the moves of the protein residues and those of the ligand are separated.
In each attempted movement, the conformations of the selected side chains (from the protein/ligand) are either changed by random perturbations or assigned by the randomly selected rotamers from a library.
For an attempt with a random perturbation, the displacement of each dihedral angle is the sum of a large rotation (N times 60 degrees with N as a random integer between 0 and 5) and a random perturbation from 0 to 30 degrees.
For a rotamer library attempt, a side-chain conformation is updated with a random rotamer from a high resolution side-chain library for protein residues \cite{xiang2001extending}, and from a homogeneous library at 10 degree resolution for the ligand.
If a structure with tolerable overlaps is generated in an attempt, it is minimized and sent to subsequent stages for judgment of acceptance.
Each side-chain move takes less than 15 seconds and is the fastest among all the three move types.

% from mike's
For side chain Monte Carlo, a steric screen with an overlap factor of 0.6 was used.
Rotamer torsional moves were selected 75\% of the time, with half of the remaining being of random torsions, and the other half random perturbations of all torsions within the randomly selected side chains.
Clusters of size 1 (i.e. single side chains), size 2 and size three were selected in equal proportion, and all side chains in the cluster were perturbed with the selected torsion move.
A mutual beta carbon distance of 6 Angstroms was used for the clustering size.
Small torsion perturbations made +/- 60 degress from the current dihedral angle, and were performed 5\% of the time; Large periodic moves were performed 95\% of the time.
Only outer steps were performed, and each side chain Monte Carlo series consisted in only one move.
Minimization was performed after the single step, and acceptance was performed at 1 K.

